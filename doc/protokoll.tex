\documentclass [] {scrartcl}
\usepackage[ngerman]{babel}
\usepackage{pgf}
\usepackage{tikz}
\usetikzlibrary{arrows,automata}
\begin{document}

  \subject{Praktikums Programmierbare Schaltkreise}
  \title{Protokoll}
  \author{Franz Gregor, Mat.Nr. 123123123 \and Studiengang Diplom-Informatik(2008)}

  \maketitle

  \section{7 Segmentanzeige}
  \subsection{4 Bit Bin"arzahl zu 7 Segment-Darstellung}
  \subsubsection{Auswertung}
  \paragraph{Ressourcenbedarf}
  \begin{table}
    \begin{tabular}{lccc}
      Implementierung & Macrozellen & Produkt Terme & Funktionsbl"ocke\\
      case statement & 12 & 20 & 1, 3, 11\\
      select statement & 12 & 20 & 1, 3, 11\\
    \end{tabular}
  \end{table}
  \subsection{3 Bit Bin"arzahl zu One-Hot-Code}
  \subsubsection{Vorbereitungen}
  \paragraph{Vorbereitungsaufgabe a)}
  Wie viele Stellen hat der One-Hot-Code, wenn genau alle Eingangsbelegungen abgedeckt werden sollen?

  Der One-Hot-Code hat $2^3 = 8$ Stellen.
  \subsubsection{Auswertung}
  \paragraph{Ressourcenbedarf}
  \begin{table}
    \begin{tabular}{ccc}
      Macrozellen & Produkt Terme & Funktionsbl"ocke\\
      8 & 8 & 5, 6, 7\\
    \end{tabular}
  \end{table}
  \subsection{21-stelliger Taktteiler}
  \subsubsection{Vorbereitungen}
  \paragraph{Vorbereitungsaufgaben a)}
  Mit welcher Frequenz blinken die LEDs LD1 bis LD8?

  LD1 blinkt mit $1,842*10^6/2^{20} = 1,76$ Hz, LD2 mit $1,842*10^6/2^{19} = 3,51$ Hz, LD3 mit $1,842*10^6/2^{18} = 7,03$ Hz,  LD4 mit $1,842*10^6/2^{17} = 14,05$ Hz, LD5 mit $1,842*10^6/2^{16} = 28,11$ Hz, LD6 mit $1,842*10^6/2^{15} = 56,21$ Hz, LD7 mit $1,842*10^6/2^{14} = 112,43$ Hz und LD8 mit $1,842*10^6/2^{13} = 224,85$ Hz.
  
  \subsubsection{Auswertung}
  \paragraph{Ressourcenbedarf}
  \begin{table}
    \begin{tabular}{ccc}
      Macrozellen & Produkt Terme & Funktionsbl"ocke\\
      21 & 24 & 1, 5, 6, 7\\
    \end{tabular}
  \end{table}
  \paragraph{Taktfrequenz der Schaltwerke}
  Das Schaltwerk kann mit maximal 256.410 MHz betrieben werden.

  \subsection{4 stellige 7 Segment Anzeige im Multiplex Betrieb}
  \subsubsection{Vorbereitungen}
  \paragraph{Vorbereitungsaufgaben a)}
  F''ur den Z''ahler des Takteils werden 7 Bit ben''otigt. Die Anode wechselt mit einem Takt von 115,125 kHz.
  \subsubsection{Besonderheiten}
  \begin{description}
    \item[Takteiler] Wird bestimmt durch die Trägheit des menschlichem Auges und der Schaltgeschwindigkeit des Auswahltransistors.
  \end{description}
  \subsubsection{Auswertung}
  \paragraph{Ressourcenbedarf}
  \begin{table}
    \begin{tabular}{ccc}
      Macrozellen & Produkt Terme & Funktionsbl"ocke\\
      18 & 17 & 1, 3, 4, 11\\
    \end{tabular}
  \end{table}
  \paragraph{Taktfrequenz der Schaltwerke}
  Das Schaltwerk kann mit maximal 256.410 MHz betrieben werden.

  \section{Codeschloss}
  \subsection{Einfuehrung}
  \subsubsection{Vorbereitungen}
  \paragraph{L"osungen der Vorbereitungsaufgaben}
  \paragraph{Statemachine}
 Die Statemachine zu dieser Aufgabe ist in Abbildung \ref{st_2_1} auf Seite \pageref{st_2_1} zu finden.
Das gleichzeitige Dr"ucken meherer Tasten wird im SM-Chart durch MORE repr"asentiert.
\begin{figure}
\begin{tikzpicture}[->,>=stealth',shorten >=1pt,auto,node distance=4cm,
                    semithick]
  \tikzstyle{every state}=[fill=red,draw=black,text=white]

  \node[state] (st1)                    {$st1$};
  \node[state]         (st2) [right of=st1] {$st2$};
  \node[initial,state]         (st0) [above of=st2] {$st0$};
  \node[state]         (st3) [right of=st2] {$st3$};

  \path
        (st0) edge [loop above]      node {MORE} (st0)
                   edge                                   node {BTN1} (st1)
                   edge                                   node {BTN2} (st2)
                   edge                                   node {BTN3} (st3)
	(st1) edge [loop left]            node {MORE or BTN1} (st1)
                   edge [bend left]                                 node {BTN2} (st2)
                   edge [bend right=95]                                 node {BTN3} (st3)
	(st2) edge [loop below]      node{MORE or BTN2} (st2)
                   edge [bend left]                                 node {BTN1} (st1)
                   edge [bend left]                                  node {BTN3} (st3)
        (st3) edge [loop right]         node {MORE or BTN3} (st3)
                   edge [bend left=55]                                 node[anchor=south] {BTN1} (st1)
                   edge [bend left]                                 node {BTN2} (st2)
;
\end{tikzpicture}
\caption{Statemachine f"ur Komplex 2 Aufgabe 1}
\label{st_2_1}
\end{figure}
\begin{figure}
\begin{tikzpicture}[->,>=stealth',shorten >=1pt,auto,node distance=4cm,
                    semithick]
  \tikzstyle{every state}=[fill=red,draw=black,text=white]

  \node[state] (st1)                    {$st1$};
  \node[state]         (st2) [right of=st1] {$st2$};
  \node[initial,state]         (st0) [above of=st2] {$st0$};
  \node[state]         (st3) [right of=st2] {$st3$};

  \path
        (st0) edge [loop above]      node {MORE or RES} (st0)
                   edge                                   node[left] {BTN1} (st1)
                   edge [bend right]                               node[left] {BTN2} (st2)
                   edge                                  node[right] {BTN3} (st3)
	(st1) edge [loop left]            node {MORE or BTN1} (st1)
                   edge [bend left]                                 node {BTN2} (st2)
                   edge [bend right=95]                                 node {BTN3} (st3)
                   edge [bend left]        node[left]{RES} (st0)
	(st2) edge [loop below]      node{MORE or BTN2} (st2)
                   edge [bend left]                                 node {BTN1} (st1)
                   edge [bend left]                                  node {BTN3} (st3)
                   edge [bend right]                             node[right]{RES} (st0)
        (st3) edge [loop right]         node {MORE or BTN3} (st3)
                   edge [bend left=55]                                 node[anchor=south] {BTN1} (st1)
                   edge [bend left]                                 node {BTN2} (st2)
                   edge [bend right]                                   node[right]{RES} (st0)
;
\end{tikzpicture}
\caption{Erweiterte Statemachine f"ur Komplex 2 Aufgabe 1}
\label{st_2_2}
\end{figure}
  \subsubsection{Besonderheiten}
  \begin{description}
    \item[Takteiler] Wodurch vorgegeben?
    \item[Erkennen mehrmaligen Dr"uckens einer Taste] Warum wichtig?
  \end{description}
  \subsubsection{Auswertung}
  \paragraph{Ressourcenbedarf}
  \begin{table}
    \begin{tabular}{lccc}
      Version & Macrozellen & Produkt Terme & Funktionsbl"ocke\\
      Teilaufgabe c & 5 & 9 & 1, 6\\
      Teilaufgabe d One-Hot & 5 & 9 & 1, 6\\
      Teilaufgabe d Gray-Code & 5 & 9 & 1, 6\\
      Teilaufgabe d Bin"arcode & 5 & 9 & 1, 6\\
      Teilaufgabe e One-Hot & 5 & 11 & 1, 6\\
      Teilaufgabe e Gray-Code & 5 & 11 & 1, 6\\
      Teilaufgabe e Bin"arcode & 5 & 11 & 1, 6\\
      Teilaufgabe f One-Hot & 5 & 11 & 1, 6\\
      Teilaufgabe f Gray-Code & 5 & 11 & 1, 6\\
      Teilaufgabe f Bin"arcode & 5 & 11 & 1, 6\\
    \end{tabular}
  \end{table}
  \subsection{Akzeptor}
  \subsubsection{Vorbereitungen}
  \paragraph{L"osungen der Vorbereitungsaufgaben}
  \begin{itemize}
    \item[Welche Eingabe muss erfolgen, damit Zustand st1, st2 sowie st3 erreicht werden?] 
    Um Zustand st1 zu erreichen muss beliebig oft BTN1 gedr"uckt werden.
    Zustand st2 wird nur von Zustand st2 aus erreicht und zwar indem beliebig oft BTN2 gedr"uckt wird.
    Zustand st3 wird wiederum nur erreicht wenn vorher st2 erreicht wurde und anschließend beliebig oft BTN3 gedr"uckt wird.
    \item[Bei welchen Eingaben wird Zustand st4 erreicht?]
    Bei allen Eingaben die nicht mit BTN1 starten oder bei dennen nach einem BTNi nicht BTNi oder BTN(i+1) gedr"uckt wird und anschließend kein RES vorkommt.
   \item[Welche Bedeutung besitzen die Zustande st3 und st4 im Sinne eines Akzeptors?]
   Zustand st3 repr"asentiert das akzeptieren der Eingabe und st4 das ablehnen.
  \end{itemize}
  \paragraph{StateMachines}
\begin{figure}
\begin{tikzpicture}[->,>=stealth',shorten >=1pt,auto,node distance=4cm,
                    semithick]
  \tikzstyle{every state}=[fill=red,draw=black,text=white]

  \node[initial,state] (st0)                    {$st0$};
  \node[state]         (st1) [below of=st0] {$st1$};
  \node[state]         (st2) [right of=st1] {$st2$};
  \node[state]         (st3) [right of=st2] {$st3$};
  \node[state]         (st4) [below of=st1] {$st4$};

  \path
        (st0) edge [loop above]      node {MORE or RES} (st0)
                   edge                                    node[left] {BTN1} (st1)
                   edge [bend right=60]         node[left]{BTN2 or BTN3} (st4)
	(st1.east) edge [bend left=90]            node{MORE or BTN1} (st1.south)
        (st1) edge                                    node {BTN2} (st2)
                   edge                                    node {BTN3} (st4)
                   edge [bend right]         node[right] {RES} (st0)
	(st2.east) edge[bend left=90]      node {MORE or BTN2} (st2.south)
        (st2)           edge                                    node {BTN3} (st3)
                   edge [bend left]                                   node {BTN1} (st4)
                   edge [bend right]           node[right] {RES} (st0)
        (st3) edge [loop below]         node {MORE or BTN3 or BTN2 or BTN1} (st3)
                   edge [bend right]           node[right] {RES} (st0)
        (st4) edge [bend left=50]             node[right] {RES} (st0)
                   edge [loop below]         node {MORE or BTN3 or BTN2 or BTN1} (st4)
;
\end{tikzpicture}
\caption{Statemachine f"ur Komplex 2 Aufgabe 2}
\label{st_2_2}
\end{figure}
  \subsubsection{Besonderheiten}
  \begin{description}
    \item[Takteiler] Wodurch vorgegeben?
    \item[Erkennen mehrmaligen Dr"uckens einer Taste] Warum wichtig?
  \end{description}
  \subsubsection{Auswertung}
  \paragraph{Ressourcenbedarf}
  \begin{table}
    \begin{tabular}{lccc}
      Version & Macrozellen & Produkt Terme & Funktionsbl"ocke\\
      Teilaufgabe c & 5 & 13 & 1, 6\\
    \end{tabular}
  \end{table}

  \subsection{Codeschloss}
  \subsubsection{Vorbereitungen}
  \paragraph{L"osungen der Vorbereitungsaufgaben}
  \paragraph{StateMachines}
\begin{figure}
\begin{tikzpicture}[->,>=stealth',shorten >=1pt,auto,node distance=2.8cm,
                    semithick]
  \tikzstyle{every state}=[fill=red,draw=black,text=white]

  \node[initial,state] (start)                    {$start$};
  \node[state]         (st1) [right of=start] {$st1$};
  \node[state]         (st2) [right of=st1] {$st2$};
  \node[state]         (st3) [right of=st2] {$st3$};
  \node[state]         (st4) [right of=st3] {$st4$};
  \node[state]         (open) [above of=st2] {$open$};
  \node[state]         (locked) [below of=st2] {$locked$};

  \path
        (start) edge node[below]{BTN1} (st1)
        (start) edge[bend right] node[above right]{OTHERS} (locked)
        (st1) edge node[below]{BTN2} (st2)
                   edge[bend right] node{RES} (start)
                   edge node[left, near start]{OTHERS} (locked)
        (st2) edge node[below]{BTN3} (st3)
                   edge[bend right] node{RES} (start)
                   edge node[near start]{OTHERS} (locked)
        (st3) edge node[below]{BTN4} (st4)
                   edge[bend right] node{RES} (start)
                   edge node[near start]{OTHERS} (locked)
        (st4) edge[bend right] node{BTN5} (open)
                   edge[bend right] node{RES} (start)
                   edge[bend left] node[right, near start]{OTHERS} (locked)
        (open) edge[bend right] node{RES} (start)
                   edge[loop above] node{OTHERS} (open)
        (locked) edge[bend left=45] node[near end]{RES} (start)
;
\end{tikzpicture}
\caption{Statemachine 1 des Codeschlosses (Akzeptor)}
\end{figure}
  \subsubsection{Besonderheiten}
  \begin{description}
    \item[Takteiler] Wodurch vorgegeben?
    \item[Erkennen mehrmaligen Dr"uckens einer Taste] Warum wichtig?
  \end{description}
  \subsubsection{Auswertung}
  \paragraph{Ressourcenbedarf}
%  \begin{table}
%    \begin{tabular}{}
    
%    \end{tabular}
%  \end{table}
  \paragraph{Taktfrequenz der Schaltwerke}
%   \begin{table}
%    \begin{tabular}{}
    
%    \end{tabular}
%  \end{table} 


\end{document}
