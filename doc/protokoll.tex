\documentclass [] {scrartcl}
\usepackage[ngerman]{babel}
\begin{document}

  \subject{Praktikums Programmierbare Schaltkreise}
  \title{Protokoll}
  \author{Franz Gregor, Mat.Nr. 123123123 \and Studiengang Diplom-Informatik(2008)}

  \maketitle

  \section{7 Segmentanzeige}
  \subsection{4 Bit Bin"arzahl zu 7 Segment-Darstellung}
  \subsubsection{Auswertung}
  \paragraph{Ressourcenbedarf}
  \begin{table}
    \begin{tabular}{lccc}
      Implementierung & Macrozellen & Produkt Terme & Funktionsbl"ocke\\
      case statement & 12 & 20 & 1, 3, 11\\
      select statement & 12 & 20 & 1, 3, 11\\
    \end{tabular}
  \end{table}
  \subsection{3 Bit Bin"arzahl zu One-Hot-Code}
  \subsubsection{Vorbereitungen}
  \paragraph{Vorbereitungsaufgabe a)}
  Wie viele Stellen hat der One-Hot-Code, wenn genau alle Eingangsbelegungen abgedeckt werden sollen?

  Der One-Hot-Code hat $2^3 = 8$ Stellen.
  \subsubsection{Auswertung}
  \paragraph{Ressourcenbedarf}
  \begin{table}
    \begin{tabular}{ccc}
      Macrozellen & Produkt Terme & Funktionsbl"ocke\\
      8 & 8 & 5, 6, 7\\
    \end{tabular}
  \end{table}
  \subsection{21-stelliger Taktteiler}
  \subsubsection{Vorbereitungen}
  \paragraph{Vorbereitungsaufgaben a)}
  Mit welcher Frequenz blinken die LEDs LD1 bis LD8?

  LD1 blinkt mit $1,842*10^6/2^{20} = 1,76$ Hz, LD2 mit $1,842*10^6/2^{19} = 3,51$ Hz, LD3 mit $1,842*10^6/2^{18} = 7,03$ Hz,  LD4 mit $1,842*10^6/2^{17} = 14,05$ Hz, LD5 mit $1,842*10^6/2^{16} = 28,11$ Hz, LD6 mit $1,842*10^6/2^{15} = 56,21$ Hz, LD7 mit $1,842*10^6/2^{14} = 112,43$ Hz und LD8 mit $1,842*10^6/2^{13} = 224,85$ Hz.
  
  \subsubsection{Auswertung}
  \paragraph{Ressourcenbedarf}
  \begin{table}
    \begin{tabular}{ccc}
      Macrozellen & Produkt Terme & Funktionsbl"ocke\\
      21 & 24 & 1, 5, 6, 7\\
    \end{tabular}
  \end{table}
  \paragraph{Taktfrequenz der Schaltwerke}
  Das Schaltwerk kann mit maximal 256.410 MHz betrieben werden.


  \section{Codeschloss}
  \subsection{4 Bit Bin"arzahl zu 7 Segment-Darstellung}
  \subsubsection{Vorbereitungen}
  \paragraph{L"osungen der Vorbereitungsaufgaben}
  \paragraph{StateMachines}
  \subsubsection{Besonderheiten}
  \begin{description}
    \item[Takteiler] Wodurch vorgegeben?
    \item[Erkennen mehrmaligen Dr"uckens einer Taste] Warum wichtig?
  \end{description}
  \subsubsection{Auswertung}
  \paragraph{Ressourcenbedarf}
%  \begin{table}
%    \begin{tabular}{}
    
%    \end{tabular}
%  \end{table}
  \paragraph{Taktfrequenz der Schaltwerke}
%   \begin{table}
%    \begin{tabular}{}
    
%    \end{tabular}
%  \end{table} 


\end{document}
